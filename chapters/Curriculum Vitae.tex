\ifmaster
  \chapter*{作者简历及攻读硕士学位期间取得的研究成果}
  \markboth{作者简历及攻读硕士学位期间取得的研究成果}{}
  \addcontentsline{toc}{chapter}{作者简历及攻读硕士学位期间取得的研究成果}
\else
  \chapter*{作者简历及攻读博士学位期间取得的研究成果}
  \markboth{作者简历及攻读博士学位期间取得的研究成果}{}
  \addcontentsline{toc}{chapter}{作者简历及攻读博士学位期间取得的研究成果}
\fi


\textbf{一、作者简历}

Zeto YEUNG,X,中国人,出生于千禧年。

\vspace{-10pt}

\begin{table}[H]
  \centering
  \renewcommand\arraystretch{0.8} % 行距
  \begin{tabular}{llll}
    2022.09$\sim$至今    & 北京交通大学 & 机械与电子控制工程学院 & 硕士研究生 \\
    2022.09$\sim$NOW    & BJTU & RRC, MECE & Master Student \\
  \end{tabular}
\end{table}

\textbf{二、发表论文}

\begin{enumerate}[label={[}\arabic*{]}]
	\item Zeto YEUNG. Zeto YEUNG的硕士生涯[Z]. 北京: 我家出版社, 2022: 22.
\end{enumerate}

\vspace{18pt}

\textbf{三、参与项目}

\begin{enumerate}[label={[}\arabic*{]}]
	\item 我家社会科学基金“面上”项目:《A Brown Fox Jumps over a Lazy Dog》,M2500000000
\end{enumerate}

\vspace{18pt}

\textbf{四、发明专利}

\begin{enumerate}[label={[}\arabic*{]}]
	\item Zeto YEUNG. 22$\sim$24岁的Zeto YEUNG: 我家,MH00222324[P].2025-02-25.
\end{enumerate}

\ifdoctor
  \chapter*{答辩决议书}
  \markboth{答辩决议书}{}
  \addcontentsline{toc}{chapter}{答辩决议书}

X士学位论文“XXXXXXX”针对XXXXXXX问题(需求),开展了XXXXXXX研究,选题具有一定的理论意义和工程应用价值。论文取得以下主要研究成果:

1.	XXXXXXX。

2.	XXXXXXX。

3.	XXXXXXX。

论文内容充实、写作规范、条理清晰、文笔流畅,表明作者已掌握本学科较坚实的基础理论和较深入的专门知识,具有一定的独立从事科研工作的能力。答辩过程中表述清楚,回答问题基本正确。

经答辩委员会无记名投票表决,一致同意通过论文答辩,并建议授予XXX同学X学X士学位。





  
\fi