\chapter*{摘要}
\markboth{摘要}{}
摘要是论文内容的高度概括,应具有独立性和自含性,即不阅读论文的全文,
就能通过摘要获得必要的信息。摘要应包括研究目的、内容、方法、结果和结论
等,重点是结果和结论。 

摘要的内容要完整、客观、准确,应做到不遗漏、不拔高、不添加。摘要应
按层次逐段简要写出,摘要在叙述研究内容、研究方法和主要结论时,除作者的
价值和经验判断可以使用第一人称外,一般使用第三人称,采用“分析了……原
因”、“研究了……”、“对……进行了探讨”“给出了……结论”等记述方法进行描
述。避免主观性的评价意见,避免对背景、目的、意义、概念和一般性(常识性)
理论叙述过多。 

摘要需采用规范的名词术语(包括地名、机构名和人名)。对个别新术语或
无中文译文的术语,可用外文或在中文译文后加括号注明外文。摘要中应尽量避
免使用图、表、化学结构式、非公知公用的符号与术语,不标注引用文献编号。 
博士学位论文摘要应包括以下几个方面的内容: 

(1)论文的研究背景及目的。简洁准确地交代论文的研究背景与意义、相
关领域的研究现状、论文所针对的关键科学问题,使读者把握论文选题的必要性
和重要性。此部分介绍不宜写得过多,一般不多于400字。 

(2)论文的主要研究方法与研究内容。介绍论文所要解决核心问题开展的
主要研究工作以及研究方法或研究手段,使读者可以了解论文的研究思路、研究
方案、研究方法或手段的合理性与先进性。 

(3)论文的主要创新成果。简要阐述论文的新思想、新观点、新技术、新
方法、新结论等主要信息,使读者可以了解论文的创新性。创新成果注意凝练和
综合,一般以2~4项为宜。 

(4)论文成果的理论和实际意义。客观、简要地介绍论文成果的理论和实
际意义,使读者可以快速获得论文的学术价值。 

摘要的字数(以汉字计),硕士学位论文一般为500~1000字,博士学位论文
一般为1000~2000字,摘要页不需写出论文题目。 

英文摘要与中文摘要的内容应完全一致,在语法、用词上应准确无误,语言
简练通顺。 

留学生的英文版硕士学位论文应有不少于3000字的“详细中文摘要”,博士学
位论文中应有不少于5000字的“详细中文摘要”。 

关键词是供检索用的主题词条,用显著的字符另起一行,排在摘要的下方。
关键词应集中体现论文特色,具有语义性,在论文中有明确的出处,并应尽量采
用《汉语主题词表》或各专业主题词表提供的规范词。每篇论文应选取3~8个关
键词。

{\bfseries 关键词:}[关键词1;关键词2;关键词3;关键词4] 




\chapter*{ABSTRACT}
\markboth{ABSTRACT}{}
The word count of the abstract (in Chinese characters) is generally 500-1000 words for master's dissertations and 1000-2000 words for doctoral dissertations.
The abstract page should not contain the title of the dissertation.

The content of the English abstract and the Chinese abstract should be identical, and the grammar and wording should be accurate, and the language should be concise and fluent.
The English abstract should be identical with the Chinese abstract in terms of grammar and diction, and the language should be concise and fluent.

The English version of the master's thesis of international students should have a detailed Chinese abstract of not less than 3000 words, and the doctoral thesis should have a detailed Chinese abstract of not less than 5000 words.
The dissertation of doctoral degree should have a "detailed Chinese abstract" of not less than 5000 words.

Keywords are the subject entries for searching, which should be placed in a separate line with prominent characters below the abstract.
Keywords should focus on the characteristics of the dissertation, be semantic, have a clear source in the dissertation, and try to use the "Chinese Theme Word List" or the "Chinese Theme Word List" as much as possible.

The keywords should focus on the characteristics of the dissertation, be semantic, have clear sources in the dissertation, and try to use the standardized words provided by Chinese Theme Word List or the theme word lists of each specialty.Each paper should choose 3-8 keywords.
Each thesis should choose 3-8 key words.


{\bfseries KEYWORDS: }[Keyword 1; Keyword 2; Keyword 3; Keyword 4]




\chapter*{序言(若有)}
\markboth{序言}{}
(可根据需要撰写或移除)学位论文的序、序言或前言,一般是作者或他人对本篇论文基本特征的简介,如说明研究工作缘起、背景、主旨、目的、意义、编写体例,以及资助、支持、协作经过等;也可以评述和对相关问题发表意见。这些内容也可以在正文绪论(引言)中说明。







