\chapter{{\protect\\} \hspace{-13pt}{[补充内容]}}

本章内容参考《清华大学学位论文\LaTeX{}模板使用示例文档\cite{TUN2025LaTeXThesisTe}v7.5.1》。

附录是与论文内容密切相关、但编入正文又影响整篇论文编排的条理和逻辑
性的资料,例如某些重要的数据表格、计算程序、统计表等,是论文主体的补充内
容,可根据需要设置。

附录中的图、表、数学表达式、参考文献等另行编序号,与正文分开,一律用
阿拉伯数字编码,但在数码前冠以附录的序号,例如“图A-1”,“表A-1”,“式
(A-1)”等。

\section{插图}
如\figref{fig:app}所示,这是一张例图。

\begin{figure}[htbp]
  \centering
  \includegraphics[width=0.8\linewidth]{bjtu_rrc.pdf}
  \bifigcaption{附录中的图片示例}{Example Image in Appendix}
  \label{fig:app}
\end{figure}

\noindent 例图结束。

\section{表格}
如\tabref{tab:app}所示,这是一张例表。

\begin{table}[H]\wuhao
  \centering
  \renewcommand\arraystretch{0.8} % 行距
  \bitabcaption{附录中的表格示例}{Example Table in Appendix}
  \begin{tabular}{ll}
    \toprule
    文件名          & 描述                         \\
    \midrule
    bjtumaster.cls   & 模板文件                     \\
    main.tex & 论文主文档    \\
    chapters & 论文章节存放目录  \\
    figures & 论文插图存放目录        \\
    betterbib.bib & 论文参考文献库(BibLaTeX)        \\
    gbt7714-numerical.bst & BibTeX 参考文献表国标样式文件    \\
    bibspacing.sty & 参考文献间距调整宏包  \\
    \bottomrule
  \end{tabular}
  \label{tab:app}
\end{table}

\section{数学表达式}

\begin{equation}\label{示例}
    M=d\left( n-g-1 \right) +\sum_{k=1}^g{f_k}+v-\xi 
\end{equation}

\equref{示例}是一个示例数学表达式,式中$d$为机构维数,$n$为构件数目,$g$为运动副数目,$\sum_{k=1}^g{f_k}$为各运动副自由度之和,$\xi$为局部自由度。


