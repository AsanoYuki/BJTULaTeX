\chapter{绪论}
\label{cha:intro}
\section{研究背景}
在学术论文写作过程中,排版规范和文档格式的标准化至关重要。现代科研工作者普遍采用 \LaTeX{} 作为主要的论文排版工具,尤其是在数学、物理、工程及计算机科学等领域。与传统的 WYSIWYG(所见即所得)编辑方式不同,\LaTeX{} 采用标记语言,能够实现高质量的公式排版、自动化参考文献管理、跨平台兼容性等优势 \cite{Lam1986LATEXDocument}。

近年来,云端协作工具的兴起进一步推动了 \LaTeX{} 在学术界的应用。Overleaf 作为当前最受欢迎的在线 \LaTeX{} 编辑器之一,为用户提供了多人协作、云端编译、版本管理等功能,极大提升了 \LaTeX{} 文档的可用性和易用性 \cite{OveOverleafZaiXianDeLaT}。

\textbf{本校研究生院提供的\LaTeX{}模板\cite{BeiJingJiao2014BeiJingJiaoTongDaXueBoShiShuoShiXueWeiLunWen}在\overleaf{}环境下几乎为不可用的状态},其版本基于清华大学\LaTeX{}模板,版本较为陈旧,有较多过时排版代码,笔者经过若干天勘误仍无法满足正常使用需求,故从0.5开始(以苯人基于「南开大学程明明老师制作的CVPR中文模板\cite{ChengMingMing2016ZhongWenMoBanMyCVPR}」制作的「东华大学本科学位论文模板」为基础),参考清华大学、浙江大学学位论文\cite{TUN2025LaTeXThesisTe, Wan2025ZheJiangDaXueXueWeiLunWenLaTeX}模板,重新制作了本模板。\textbf{本模板仅在\overleaf{}平台\XeTeX{} 2024 编译环境下进行测试,可供二次修改并传播。本模板遵守遵守 \LaTeX{} Project Public License,使用前请认真阅读协议内容;遵从BY-NC-SA(署名-非商业性使用-相同方式共享):使用者可以对本创作进行转载、节选、混编、二次创作,但不得运用于商业目的,且使用时须进行署名,采用本创作的内容必须同样采用本协议进行授权。作为学位论文提交时可免除署名,但可以在致谢中提及。}

\section{研究现状}
针对 \LaTeX{} 论文模板的开发,各高校和学术机构纷纷制定了符合自身格式要求的 \LaTeX{} 论文模板。例如,美国麻省理工学院(MIT) 和 加州大学伯克利分校(UC Berkeley) 分别发布了适用于本校硕博士学位论文的 \LaTeX{} 模板 \cite{MasMITthesistemp, PauPreparingYour}。在国内,清华大学、浙江大学等高校也提供了官方或非官方的 \LaTeX{} 论文模板 \cite{TUN2025LaTeXThesisTe, Wan2025ZheJiangDaXueXueWeiLunWenLaTeX}。然而,部分高校的 \LaTeX{} 论文模板更新较慢,且对 Overleaf 适配性不足,使得部分研究生仍需进行手动调整。

\section{研究目标与意义}
本研究旨在开发一款\textbf{适用于 Overleaf 平台}的《\textbf{北京交通大学学术硕士学位论文 \LaTeX{} 模板 ver.2025}》,满足学校对硕士学位论文的\textbf{格式规范},并针对 Overleaf 的特性进行优化。模板的主要目标包括:
\begin{enumerate}[itemsep=0pt,topsep=2pt,parsep=0pt]
    \item \textbf{符合北京交通大学硕士学位论文排版规范},包括封面、目录、章节标题、参考文献等格式要求;
    \item \textbf{兼容 Overleaf 云端编译环境},简化用户操作,降低使用门槛;
    \item \textbf{提供详细的模板使用说明},涵盖文档结构、数学公式、表格、插图、参考文献管理等内容,以便研究生能够快速上手并专注于论文撰写。
\end{enumerate}

本模板的开发不仅能够提高\textbf{论文排版的标准化},也能帮助学术工作者减少\textbf{格式调整}的时间成本,使其更加专注于研究本身。

[注]\quad 本章内容部分由ChatGPT 4o生成,请谨慎甄别。



