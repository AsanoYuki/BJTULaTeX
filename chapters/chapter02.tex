
\chapter{论文撰写结构与规范}
\label{cha:2}

学位论文基本结构包括 5 个组成部分:前置部份、主体部份、参考文献、附
录和结尾部份。

\section{前置部份}
主要包含:
\begin{enumerate}[itemsep=0pt,topsep=2pt,parsep=0pt]
  \item 封面
  \item 学位论文版权使用授权书
  \item 题名页
  \item 致谢
  \item 摘要页
  \item 英文摘要页
  \item 序言或前言(可根据需要)
  \item 目录
  \item 图和附表清单(可根据需要)
  \item 符号、标志、缩略词、首字母缩写、计量单位、术语等的注释表(可根据需要)
\end{enumerate}

\subsection{题名 }

题名应以简明的词语,恰当、准确、科学地反映论文最重要的特定内容(一
般不超过 25 字),应中英文对照。题名通常由名词性短语构成,应尽量避免使用
不常用的缩略词、首字母缩写字、字符、代号和公式等。 

如题名内容层次很多,难以简化时,可采用题名和副题名相结合的方法,副
题名起补充、阐明题名的作用。中文的题名与副题名之间用破折号相连,英文则
用冒号相连。题名和副题名在整篇学位论文中的不同地方出现时,应保持一致。 

\subsection{致谢}

放置在摘要页前,对象包括:
\begin{enumerate}[itemsep=0pt,topsep=2pt,parsep=0pt]
  \item 国家科学基金,资助研究工作的奖学金基金,合同单位,资助或支持的企业、组织或个人。
  \item 协助完成研究工作和提供便利条件的组织或个人。
  \item 在研究工作中提出建议和提供帮助的人。
  \item 给予转载和引用权的资料、图片、文献、研究思想和设想的所有者。
  \item 其他应感谢的组织和个人。 
\end{enumerate}

\subsection{摘要与关键词}
摘要是论文内容的高度概括,应具有独立性和自含性,即不阅读论文的全文,
就能通过摘要获得必要的信息。摘要应包括研究目的、内容、方法、结果和结论
等,重点是结果和结论。 

摘要的内容要完整、客观、准确,应做到不遗漏、不拔高、不添加。摘要应
按层次逐段简要写出,摘要在叙述研究内容、研究方法和主要结论时,除作者的
价值和经验判断可以使用第一人称外,一般使用第三人称,采用“分析了……原
因”、“研究了……”、“对……进行了探讨”“给出了……结论”等记述方法进行描
述。避免主观性的评价意见,避免对背景、目的、意义、概念和一般性(常识性)
理论叙述过多。 

摘要需采用规范的名词术语(包括地名、机构名和人名)。对个别新术语或
无中文译文的术语,可用外文或在中文译文后加括号注明外文。摘要中应尽量避
免使用图、表、化学结构式、非公知公用的符号与术语,不标注引用文献编号。 
博士学位论文摘要应包括以下几个方面的内容: 

(1)论文的研究背景及目的。简洁准确地交代论文的研究背景与意义、相
关领域的研究现状、论文所针对的关键科学问题,使读者把握论文选题的必要性
和重要性。此部分介绍不宜写得过多,一般不多于400字。 

(2)论文的主要研究方法与研究内容。介绍论文所要解决核心问题开展的
主要研究工作以及研究方法或研究手段,使读者可以了解论文的研究思路、研究
方案、研究方法或手段的合理性与先进性。 

(3)论文的主要创新成果。简要阐述论文的新思想、新观点、新技术、新
方法、新结论等主要信息,使读者可以了解论文的创新性。创新成果注意凝练和
综合,一般以2~4项为宜。 

(4)论文成果的理论和实际意义。客观、简要地介绍论文成果的理论和实
际意义,使读者可以快速获得论文的学术价值。 

摘要的字数(以汉字计),硕士学位论文一般为500~1000字,博士学位论文
一般为1000~2000字,摘要页不需写出论文题目。 

英文摘要与中文摘要的内容应完全一致,在语法、用词上应准确无误,语言
简练通顺。 

留学生的英文版硕士学位论文应有不少于3000字的“详细中文摘要”,博士学
位论文中应有不少于5000字的“详细中文摘要”。 

关键词是供检索用的主题词条,用显著的字符另起一行,排在摘要的下方。
关键词应集中体现论文特色,具有语义性,在论文中有明确的出处,并应尽量采
用《汉语主题词表》或各专业主题词表提供的规范词。每篇论文应选取3~8个关
键词。

\subsection{序言或前言(可根据需要)}
学位论文的序言或前言,一般是作者对本篇论文基本特征的简介,如说明研
究工作缘起、背景、主旨、目的、意义、编写体例,以及资助、支持、协作经过
等。这些内容也可以在正文引言(绪论)中说明。 

\subsection{目录}
论文中各章节的顺序列表,包括论文中全部章、节、条三级标题及其起始页
码。排在序言或前言之后,另起页。 

\section{主体部份}
主要包含:绪论(引言)、正文及结论等部分。


\subsection{绪论(引言)}
绪论(引言)一般作为第1章。绪论(引言)应包括:本研究课题的来源、
背景及其理论意义与实际意义,国际上与课题相关研究领域的研究进展及成果,
存在的不足或有待深入研究的问题,归纳出将要开展研究的理论分析框架、研究
内容、研究程序和方法。 

绪论(引言)部分要注意对论文所引用国内外文献的准确标注\cite{BeiJingJiao2014BeiJingJiaoTongDaXueBoShiShuoShiXueWeiLunWen}。

\subsection{正文}
正文是学位论文的主要部分,必须实事求是、客观真切、准备完备、合乎逻
辑、层次分明、简练可读。论文各章之间应该前后关联,构成一个有机的整体。
论文给出的数据必须真实可靠,推理正确,结论明确,无概念性和科学性错误。
对于科学实验、计算机仿真的条件、实验过程、仿真过程等需加以叙述,然后对
结果进行分析及理论提升,避免直接给出结果、曲线和结论。引用他人研究成果
或采用他人成果时,应注明出处,不得将其与本人提出的理论分析混淆在一起。

论文主体各章后应有一节“本章小结”,实验方法或材料等章节可不写“本
章小结”。各章小结是对各章研究内容、方法与成果的简洁准确的总结与概括,也
是论文最后结论的依据。

\subsection{结论}
论文的结论是最终的、总体的结论,不是正文中每章小结的简单重复,也不
能与摘要混为一谈。结论作为学位论文正文的组成部分,单独成章,不标注引用
文献。结论应包括论文的核心观点,交待研究工作的局限,提出未来工作的意见、
建议或展望。结论应该准确、完整、明确、精炼。 

如果不能导出一定的结论,也可以没有结论而进行必要的讨论。


\section{参考文献}
参考文献应置于正文后,并另起页。 

所有被引用文献均要列入参考文献中,必须按顺序标注,但同一篇文章只用
一个序号。 

尽量引用原始文献。当不能引用原始文献时,要将二次引用文献、原始文献
同时标注。 

博士学位论文的参考文献一般不少于 100 篇,学术型硕士学位论文的参考文
献一般不少于50篇,专业硕士学位论文的参考文献一般不少于30篇,其中外文
文献一般不少于总数的1/2。参考文献中近五年的文献数一般应不少于总数的1/3,
并应有近两年的参考文献和一定数量的学位论文或专业名著。 

产品说明书、未公开发表的研究报告(著名的内部报告如PB、AD报告及著名
大公司的企业技术报告等除外)以及其它无法通过公开途径获得的文献资料通常
不宜作为参考文献引用。 

引用网上参考文献时,应注明该文献的准确网页地址,网上参考文献和各类
标准不包含在上述规定的文献数量之内。本人在攻读学位期间发表的学术论文不
应列入参考文献中。

\section{附录(可根据需要)}
附录作为主体部分的补充,可包括以下内容: 

——为了整篇论文材料的完整,但编入正文又有损于编排的条理性和逻辑性,
这一材料包括比正文更为详尽的信息、研究方法和对技术更深入的叙述,对了解
正文内容有用的补充信息等。 

——由于篇幅过大或取材于复制品而不便于编入正文的材料。

——不便于编入正文的罕见珍贵资料。 

——对一般读者并非必要阅读,但对本专业同行有参考价值的资料。 

——正文中未被引用但被阅读或具有补充信息的文献。 

——某些重要的原始数据、数学推导、结构图、统计表、计算机打印输出件等。 

\section{结尾部分}
放在附录后,对象包括:
\begin{enumerate}[itemsep=0pt,topsep=2pt,parsep=0pt]
  \item {索引(可根据需要)}
  \item {作者简历及攻读学位期间取得的研究成果} 
  \item {独创性声明} 
  \item {学位论文数据集} 
  \item {封底}
\end{enumerate}

\subsection{作者简历及攻读学位期间取得的研究成果}
作者简历包括教育经历和工作经历。 

学位论文后应列出研究生在攻读学位期间发表的与学位论文内容相关的学术
论文(含已录用的论文),与学位论文无关的学术论文不宜在此列出。攻读学位期
间所获得的科研成果、专利可单做一项分别列出。

\subsection{独创性声明 }
作者可直接下载本部分内容电子版。作者本人签署姓名。

\subsection{学位论文数据集 }
由反映学位论文主要特征的数据组成,共33项: 

A1 关键词*,A2 密级*,A3 中图分类号,A4 UDC,A5 论文资助;

B1 学位授予单位名称*,B2 学位授予单位代码*,B3 学位类别*,\\B4 学位级
别*;

C1 论文题名*,C2 并列题名,C3 论文语种*;

D1 作者姓名*,D2 学号*;

E1 培养单位名称*,E2 培养单位代码*,E3 培养单位地址,E4 邮编;

F1 学科专业*,F2 研究方向*,F3 学制*,F4 学位授予年*,F5 论文提交日
期\*;

G1 导师姓名*,G2 职称*;

H1 评阅人,H2 答辩委员会主席*,H3 答辩委员会成员;

I1 电子版论文提交格式,I2 电子版论文出版(发布)者,I3 电子版论文出
版(发布)地,I4 权限声明;

J1 论文总页数*。