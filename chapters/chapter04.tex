\chapter{本地编译常见说明}

本模板本意为开发用于\overleaf 环境编译使用,但不排除有同学有本地编译需求或其它在线\LaTeX{}平台编译需求。特将几个在VSCode环境下编译报错重点以及修改方式进行说明。

\section{编译器Compiler错误导致编译失败}
本模板仅支持\XeTeX{} 编译器编译,不支持pdf\LaTeX{} 等旧时代编译器,请修改编译器设置(如通过修改VSCode的*.json文件。)

\section{字体问题}

\textbf{问题原因:}本模板采用\overleaf 平台内置的开源可商用字体。部分字体可能计算机本地未安装,因此无法编译。

\textbf{修复方法:}查看计算机已安装字体,并将\cs{bjtuthesis.cls}中的字体设置改为本地已有字体的名称。如将\cs{bjtuthesis.cls}中的280-289行从

\begin{lstlisting}[language={[LaTeX]TeX}]
    \setCJKmainfont[
  BoldFont = {Noto Serif CJK SC Bold},  % 主字体的增加字重字形
  ItalicFont = {AR PL UKai CN}  % 主字体的意大利字形
]{Noto Serif CJK SC}    % 主字体
\setCJKsansfont{Noto Sans CJK SC}   % 主无衬线字体
\setCJKmonofont{Noto Sans Mono CJK SC}  %主等宽字体
\setCJKfamilyfont{song}{Noto Serif CJK SC} % 宋体
\setCJKfamilyfont{hei}{Noto Sans CJK SC} % 黑体
\setCJKfamilyfont{fs}{cwTeXFangSong} % 仿宋
\setCJKfamilyfont{kai}{AR PL UKai CN} % 楷体
\end{lstlisting}

\noindent 直接修改为
\begin{lstlisting}[language={[LaTeX]TeX}]
    \setCJKmainfont[
  BoldFont = {宋体},  % 主字体的增加字重字形
  ItalicFont = {楷体}  % 主字体的意大利字形
]{宋体}    % 主字体
\setCJKsansfont{宋体}   % 主无衬线字体
\setCJKmonofont{黑体}  %主等宽字体
\setCJKfamilyfont{song}{宋体} % 宋体
\setCJKfamilyfont{hei}{黑体} % 黑体
\setCJKfamilyfont{fs}{仿宋} % 仿宋
\setCJKfamilyfont{kai}{楷体} % 楷体
\end{lstlisting}

可能导致显示效果不及预期,推荐使用思源宋体、更纱黑体等具有可变字重的字体以及开源可商用无版权风险的字体。(如常见的「微软雅黑」为不可商用的具有版权风险的字体。)

\section{*.svg格式的矢量图导致的编译错误}

\textbf{问题原因:}\secref{插图}一节中介绍了如何插入*.svg格式的矢量图,但其原理依旧是先将*.svg在内部转换为*.pdf再进行编译。部分编译环境如VSCode可能需要:

\begin{enumerate}[label=(\arabic*)]
    \item 修改*.json文件;
    \item 安装第三方插件;
    \item 将*.svg输出的*.pdf文件,其输出路径重定向指\cs{figure}目录在本地的绝对目录。
\end{enumerate}

\textbf{解决方法:}

方案一:建议将*.svg矢量图转换为*.pdf格式进行调用,避免直接调用*.svg矢量图。或直接转换为位图进行调用。均可正常进行编译。

方案二:对「问题原因」中所述的问题自行进行修复。